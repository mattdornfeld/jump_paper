Figure \ref{fig:jumps} shows the output of the clustering algorithm, for each rat, for the values of $\gamma$ and b that minimize the cost function, along with plots, for relevant values of n, of equation \ref{eq:freqmode2} with $\gamma$ and b equal to those same values. The values of b are not shown here, since they were all found to be effectively zero, as expected. Figure \ref{fig:jumps}(a) shows an enlarged output of the clustering algorithm for one rat. For this rat, the data is grouped into eight clusters- four that correspond to up transitions and four that correspond to down transitions. This rat exhibits a significant number of jump points in the $2\rightarrow3$ and $3\rightarrow2$ transition clusters, which is somewhat rare, since these clusters are mostly populated by jumps between constant frequency calls in the 30 kHz range to constant frequency calls in the 50 kHz and vice versa, which were rarely recorded for an unknown reason. These clusters are noticeably distinct from the others and can be separated out by visual inspection. The clusters that represent transitions between higher regions present more of an overlap with each other. The $3\rightarrow4$, $5\rightarrow6$, $4\rightarrow3$, and $6\rightarrow5$ clusters show clear drop offs in density at their borders, which seems to be well represented in the output of the algorithm. However, the $4\rightarrow5$ and $5\rightarrow4$ transition clusters do not seem to be well represented in the data for this rat. Thus, it is hard to identify these clusters based on visual inspection of the density of jump points, although we still believe them to be present in smaller numbers. It is also worth noting that, removing those clusters from the algorithm did not produce a significant change in the calculated value of $\gamma$. 

For each cluster, the ratio of after jump frequency to before jump frequency appears to be gaussianly distributed with mean approximately equal to the slope of each line that defines the cluster. Thus, the clustering algorithm accomplishes the task of separating out the overlapping density functions of each mode transition as well determining parameters for the whistle model that best fits the jump data. Figure \ref{fig:gamma_error} shows the value of $\gamma$ that minimizes equation \ref{eq:cost}, for each rat, along with the 95 \% confidence interval obtained from bootstrapping analysis. Since the jump points in Figure \ref{fig:jumps} fall along straight lines that can be separated in transition clusters, we can confidently conclude rat frequency jumps- and thus rat ultrasonic narrowband calls- are well explained by a whistle model. However, it can be seen that there is some variance in the values of $\gamma$ obtained for each rat. Furthermore, they do not exactly agree with the published value of $\gamma=0.25$ for hole tone whistles. However, this value was obtained with rigid boundaries and a jet that was perfectly aligned with the downstream aperture- both qualities we are unlikely to see in a biological system. Thus, we cannot conclude from Figure \ref{fig:gamma_error} that ultrasonic whistles use a hole tone mechanism. However, considering the other results of this paper and the geometry of the rat vocal tract, this is likely the case. To verify this biologically accurate fluid dynamic simulations would need to be performed in an attempt to reproduce frequency jumps.