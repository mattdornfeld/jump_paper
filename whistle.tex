As mentioned above, several experimental studies suggest the mechanism behind rat narrowband vocalizations is a fluid dynamic whistle. However, the type of whistle that constitutes this mechanism is still a point of conjecture. In general, fluid dynamic whistles consist of a jet of air that destabilizes into a vortex street. This vortex street interacts with downstream structures, radiating acoustic energy into the far field, and induces a velocity that provides energy to the perturbation frequencies, in the destabilized jet, which are close to the vortex shedding frequency. The perturbation frequencies with more energy become more unstable and are more likely to make contributions to the vortex street. This forms a positive feedback mechanism, which results in a tonal frequency emitted into the acoustic far field equal to the vortex shedding frequency of the fully formed vortex street. The destabilization process is quite complicated, but analytic results can be obtained by neglecting it and assuming the existence of an already fully formed vortex street.
\begin{equation}
\label{eq:whistle}
\frac{fL}{U_{c}}=n+\gamma
\end{equation}
applies well to all fluid dynamic whistles that do not involve vibration of solid boundaries. The general procedure for it's derivation consists of assuming the form of the vortex street and calculating the velocity induced by the interaction of the vortex street with the downstream structure and matching its phase to that of the perturbations in the jet to ensure the positive feedback mechanism. In equation \ref{eq:whistle}, $f$ is the tonal frequency emitted in the acoustic far field (which is also the vortex shedding frequency), $L$ is a characteristic length scale of the system, $U_c$ is the convection velocity of the vortices to which the laminar flow destabilizes into, $n$ is the acoustic mode number, and $\gamma$ is a constant that is dependent on the type of whistle. The quantity, $\frac{U_c}{f}$ is the spacing between successive vortex rings, thus $n+\gamma$ has the physical interpretation of being the number of disturbance wavelengths that can fit in the length $L$. The quantity $\gamma$ is also proportional to the phase difference $\theta$ between the acoustic feedback and the vortex formation rate.


By considering the interaction of the vortex streeta relation between the acoustic frequencies and the operating modes of the whistle can be determined by This vortex street interacts with downstream structures and induces a velocity that provides acoustic feedback to the jet, amplifying disturbance frequencies close to the vortex shedding rate and radiating sound into the acoustic far field. For jet velocities much less than the speed of sound, the equation
\begin{equation}
\label{eq:whistle}
\frac{fL}{U_{c}}=n+\gamma
\end{equation}
applies well to all fluid dynamic whistles that do not involve vibration of solid boundaries. Here $f$ is the tonal frequency emitted in the acoustic far field (which is also the vortex shedding frequency), $L$ is a characteristic length scale of the system, $U_c$ is the convection velocity of the vortices to which the laminar flow destabilizes into, $n$ is the acoustic mode number, and $\gamma$ is a constant that is dependent on the type of whistle. The quantity, $\frac{U_c}{f}$ is the spacing between successive vortex rings, thus $n+\gamma$ has the physical interpretation of being the number of disturbance wavelengths that can fit in the length $L$. The quantity $\gamma$ is also proportional to the phase difference $\theta$ between the acoustic feedback and the vortex formation rate.

As mentioned above, several experimental studies suggest the mechanism behind rat narrowband vocalizations is a fluid dynamic whistle. However, the type of whistle that constitutes this mechanism is still a point of conjecture. In the succeeding paragraphs, we will discuss several different whistle types, how they possibly relate to the anatomy of the rat vocal tract, and a quantitative prediction scheme that will allow us to determine the type used in rat vocalizations.       
In general, fluid dynamic whistles consist of a jet of air that destabilizes into a vortex street. This vortex street interacts with downstream structures and induces a velocity that provides acoustic feedback to the jet, amplifying disturbance frequencies close to the vortex shedding rate and radiating sound into the acoustic far field. For jet velocities much less than the speed of sound, the equation
\begin{equation}
\label{eq:whistle}
\frac{fL}{U_{c}}=n+\gamma
\end{equation}
applies well to all fluid dynamic whistles that do not involve vibration of solid boundaries. Here $f$ is the tonal frequency emitted in the acoustic far field (which is also the vortex shedding frequency), $L$ is a characteristic length scale of the system, $U_c$ is the convection velocity of the vortices to which the laminar flow destabilizes into, $n$ is the acoustic mode number, and $\gamma$ is a constant that is dependent on the type of whistle. The quantity, $\frac{U_c}{f}$ is the spacing between successive vortex rings, thus $n+\gamma$ has the physical interpretation of being the number of disturbance wavelengths that can fit in the length $L$. The quantity $\gamma$ is also proportional to the phase difference $\theta$ between the acoustic feedback and the vortex formation rate.

Perhaps the most widely studied whistle system is the edge tone, which consists of a plane jet that impinges on a sharp downstream edge. The interaction with the edge causes the jet to destabilize into an antisymmetric K\'{a}rm\'{a}n vortex street. The velocity induced by this vortex street then further destabilizes the jet, amplifying the perturbations with frequencies close to the vortex shedding rate, which causes more vortices to be shed at that frequency. This forms the positive feedback mechanism necessary for self sustained oscillations and the emittance of narrowband sound into the acoustic far field. By considering this feedback mechanism, equation \ref{eq:whistle} for the edge tone can be determined, with $\gamma=\frac{1}{8}+\theta$. In addition, if we consider the limit in which the perturbation wavelengths are much greater than the jet width $d$, the thin jet approximation to equation \ref{eq:whistle} can be determined. By comparing this equation with the one obtained from vortex feedback considerations, also taking into account momentum conservation in and out of the vortex formation region, we can determine $\gamma\approx0.54$. In terms of rat anatomy, it is possible the vocal folds, focus air coming up from the lungs into a plane jet which impinges on an edge like downstream structure- the most likely candidate for which is the junction between the nasal and bucaal cavities. However, if this were the mechanism, it would be expected that sound is emitted equally from both the nose and the mouth, and it has been observed that the rat ultrasonic vocalizations primarily emerge from the mouth. 

It is also possible air emerging from the vocal folds flows over the junction to the nasal cavity into the bucaal cavity. In this model, the nasal cavity acts as resonator. Vortices are shed at the leading edge and interact with the cavity and trailing edge to induce a velocity that radiates sound into the far field and provides acoustic feedback to reinforce vortex production at the leading edge. The analysis of cavity resonators depends on whether the cavity depth is much greater than the cavity width. In the deep cavity model, the frequency mode relation can be obtained by maximizing the power radiated from the cavity into the acoustic farfield. Equation \ref{eq:whistle} is found with $L$ corresponding to the width of the cavity and $\gamma\approx-0.08$. Although the length of the nasal cavity is much greater than the width of the entrance to the nasal cavity at the nasal-bucaal junction, we include the results for the shallow cavity model here for completeness. By considering the velocity induced by the vortex street in the shallow cavity. Taking the time at which a vortex is released to be when the fundamental component of the induced velocity points in to the cavity (a fact that has been verified experimentally), we find equation \ref{eq:whistle} with $\gamma\approx0.17$.      

The hole is tone what we believe to be the most biologically realistic model for our system. It consists of a circular jet impinging on a downstream aperture of slightly greater width. In this model, the vocal folds focus the air from lungs into a circular jet. The most likely anatomical structures for the downstream aperture are either the space between the base of the tongue and the soft palate or the one between the epiglottis and the cranial wall of the oropharynx. The circular jet destabilizes into a street of vortex rings axisymmteric about the jet axis. The vortex street causes air in the downstream aperture to oscillate, which radiates sound out of the mouth. The value of $\gamma$ in Figure \ref{fig:theta_error} was obtained experimentally as well as theoretically under the condition that the length between the apertures is much greater than the radius of the jet- a condition that is not necessarily met in the rat vocal tract as the radius of the opening in the vocal folds during vocalization as well as the distance between the apertures are both on the order of 1 mm \cite{Brudzynski2010,Chanaud1965,}.

Perhaps the most biologically realistic model for our system is the hole tone, which consists of a circular jet impinging on a downstream aperture of slightly greater width. In this model, the vocal folds focus the air from lungs into a circular jet. The most likely anatomical structures for the downstream aperture are either the space between the base of the tongue and the soft palate or the one between the epiglottis and the cranial wall of the oropharynx. At higher Reynolds numbers ($R_e\approx1000$), the jet destabilizes into a street of vortex rings symmetric about the jet axis. Like in the other whistle types, the interaction of the vortex rings with the downstream aperture provides the necessary feedback mechanism for sustained oscillations and the presence of tonal sound in the acoustic far field. By considering the acoustic pressure in the far field caused by the vortex street passing through a downstream circular aperture as the feedback mechanism, the frequency-mode relation of this system can be determined with $L$ being the length between the source of the jet and downstream aperture and $\gamma\approx-frac{1}{4}$. A fact that has also been determined experimentally. It is important to note, this relation was determined under the condition that the length between the source of the jet and downstream aperture is much greater than the radius of the jet. A condition that is not necessarily met in the rat vocal tract. This will be discussed more later in the analysis of the data.      