In addition, by considering the
\begin{equation}
\label{eq:edge_tone}
\frac{fL}{U_{c}}=n+\frac{1}{8}+\theta,
\end{equation}
where $n$ is the mode number, which corresponds to the number of complete distrubance wavelengths that are present in the distance between the aperture and the edge, and $2\pi\phi$ is the phase difference between the vortex formation rate and the feedback from the induced velocity.

By considering the limit in which the perturbation wavelengths are much greater than the jet width $d$, the thin jet approximation to equation \ref{eq:edge_tone} can be determined to be
\begin{equation}
\label{eq:edge_tone2}
\frac{fL}{U}=\sqrt{\frac{d}{L}},
\end{equation}
where $U$ is the mean jet velocity, not the vortex convection velocity. Comparing equations \ref{eq:edge_tone} and \ref{eq:edge_tone2} with the addition of the relation $\frac{U_c}{U}\approx0.945(\left\frac{fd}{U})\right^{1/3}$, obtained from considering the flux of momentum in and out of the vortex region, we can get a determination of the parameter $\theta\approx0.42$.