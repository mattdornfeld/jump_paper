
%might take this out
Although the rodent ultrasound production mechanism has not been widely investigated, several studies on rats suggest that these calls are actually the result of a whistle mechanism, in which air is forced through a tight aperture of constant radius, created by adduction of the vocal folds. The first of these studies was done by placing rats in a polythene bag filled with heliox gas. It was found that the presence of the heliox gas increased the fundamental frequency of the call, which is consistent with the whistle model but inconsistent with the vibrating vocal fold model, since the fundamental frequency in the vibrating model would be determined by the oscillation of the vocal folds \cite{Roberts1975b}. These results were later confirmed by injecting heliox gas into the trachea of rats while they were making ultrasonic calls and observing an instantaneous increase in fundamental frequency \cite{Riede2011a}. Furthermore, another study was done in which a camera was inserted into the tracheae of anesthetized rats while the production of an ultrasonic call was stimulated. It was observed that the vocal folds remained tense throughout the duration of a call and did not vibrate. These studies discredit the vibrating vocal fold model and indicate rat ultrasonic calls are the result of a whistle mechanism, however it is unknown what type of whistle it is \cite{Brudzynski2010Chapter}.

We performed this calculation on several different rats. Figure \ref{fig:v6_jumps} shows the results of the clustering algorithm on one such rat. It can be observed the data is grouped into eight clusters, half of which represent increases in mode number, the other half represent the corresponding decreases in mode number. Figure shows the corresponding cost function, which has a minimum at $\theta=0.43$. These results are promising for several reasons. First, the cost function has a well defined minimum. In addition, the algorithm seems to pick out, relatively well, natural high density areas of the jump points. The resulting clusters also seem to pick out global salient features of the spectrograms that are not confined to the region around the jump point. For example, the $2\rightarrow3$ and $3\rightarrow2$ clusters largely consist of transitions between 30 kHz and 50 kHz frequency unmodulated calls. On the other hand, the $5\rightarrow6$ and $6\rightarrow5$ clusters consist largely of highly frequency modulated calls that jump to a high frequency flat call then back down again. 
